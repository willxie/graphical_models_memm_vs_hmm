\section{Introduction}
[TODO:make it more motiviating]A frequent problem in many disciplines is the challenge to do sequence labelling. DNA sequencing, video semantic analysis, and part-of-speech tagging are just some of the examples where sequence labelling is being used.~\cite{dnaEx, videoEx, nlpEx}. In natural language processing (NLP), part-of-speech (POS) or lexical category tagging is an important stepping stone to solving more complex problems such as semantic analysis of sentences. The typical techniques applied to this problem are hidden Markov models (HMM), maximum entropy markov models (MEMM), and conditional random fields (CRF)~\cite{nlpBook}. While CRF is considered to be the state-of-the-art for POS tagging, [TODO: why MEMM and HMM and not CRF]. [TODO: kinda redundant for such sort paper;also talka bout implementation]This paper is structured in the following way: first we provide a brief description of the corpora used in this investigation. Then, we discuss some of the differences between HMMs and MEMMs. Next, we explain how the training process works specifically for POS tagging. Then, we describe our experimental set up. Finally, we compare the performance of HMMs vs MEMMs under similar circumstances.
